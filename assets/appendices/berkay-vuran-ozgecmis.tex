\documentclass[10pt,a4paper,sans]{moderncv}

\moderncvstyle{banking}
\moderncvcolor{blue}

\usepackage[utf8]{inputenc}
\usepackage[T1]{fontenc}
\usepackage{carlito}
\renewcommand{\familydefault}{\sfdefault}
\usepackage{microtype}
\usepackage[scale=0.85]{geometry}

% Personal data
\name{Berkay}{Vuran}
\title{Lider Ürün Analisti}
\address{Ankara, Türkiye}{}{}
\phone[mobile]{+90 542 423 99 30}
\email{berkaypsy@gmail.com}
\homepage{berkayvuran.com}
\social[linkedin]{berkay-vuran}

\begin{document}

\makecvtitle

% Professional Summary
\section{Özet}
\cvitem{}{7+ yıllık deneyime sahip ürün profesyoneli. AI, PropTech ve kurumsal yazılım sektörlerinde ürün geliştirme, kullanıcı deneyimi ve veri odaklı strateji oluşturma konularında uzman. 6+ milyon kullanıcıya hizmet eden ürünlerde gelir artışı, hız ve müşteri memnuniyeti iyileştirmeleri konusunda kanıtlanmış başarılar. PMP, PSPO, PSM, PSU ve PSD sertifikalarına sahip.}

% Experience
\section{Deneyim}
\cventry{09.2024--Günümüz}{Lider Ürün Analisti}{SESTEK}{Ankara (Hibrit)}{}{
\begin{itemize}
\item Ürün geliştirme süreçlerinde genel hız iyileştirmelerine katkı
\item Çapraz ekiplerle uyumlu çalışarak tekrar iş yükünün azaltılmasına destek
\item Yapılandırılmış geri bildirim döngüleriyle müşteri memnuniyetinde artış sağlanmasına katkı
\item Yeni ürün özelliklerinin keşif ve geliştirme süreçlerinde aktif rol
\end{itemize}}

\cventry{11.2022--09.2024}{Kıdemli Ürün Yöneticisi \& Scrum Master}{Emlakjet}{İstanbul (Uzaktan)}{}{
\begin{itemize}
\item Pazar yeri platformunda ürün yaşam döngüsünü yöneterek kullanıcı etkileşiminde artış sağlama
\item NLP destekli emlak eşleştirme dahil ML/AI entegrasyonlarıyla dönüşüm oranını iyileştirme
\item Veri destekli kararlar için KPI panosu ve analitik çerçeve oluşturma
\end{itemize}}

\cventry{01.2022--11.2022}{Ürün Sahibi}{Navlungo}{İstanbul (Uzaktan)}{}{
\begin{itemize}
\item 3 kıta ve 230 ülkede ürün gereksinimlerini yöneterek uluslararası genişlemeye katkı
\item Yüksek kullanıcı ve işlem hacmini destekleyecek şekilde platform ölçeklendirme
\end{itemize}}

\cventry{11.2021--01.2022}{Kıdemli İş Analisti}{Etiya}{Ankara (Hibrit)}{}{
\begin{itemize}
\item Lansman sonrası kusurları azaltan kapsamlı UAT çerçevesi geliştirme
\item Kurumsal müşteriler için kullanıcı araştırmaları yürütme
\end{itemize}}

\cventry{02.2019--11.2021}{Ürün Sahibi}{T.C. Sağlık Bakanlığı}{Ankara}{}{
\begin{itemize}
\item Yılda milyonlarca kullanıcıya hizmet eden ALO184 SABİM platformunun ürün geliştirme süreçlerinde görev alma
\item Buluşma Noktası Web Uygulaması’nı tasarlama ve hayata geçirme
\end{itemize}}

% Skills
\section{Yetkinlikler}
\cvitem{Ürün}{Ürün Stratejisi, Yol Haritası, Önceliklendirme, A/B Testi, Müşteri Geliştirme}
\cvitem{Analitik}{AWS QuickSight, Google Analytics, SQL, Tableau, Power BI, Mixpanel, Amplitude}
\cvitem{Araçlar}{Jira, Confluence, Figma, Miro, Azure DevOps, Notion, Productboard}
\cvitem{Metodoloji}{Agile, Scrum, Kanban, OKR, Tasarım Düşüncesi, Lean Startup}
\cvitem{Teknoloji}{API Entegrasyonu, AWS, Makine Öğrenmesi, NLP}

% Education
\section{Eğitim}
\cventry{2019--2024}{Yüksek Lisans, Örgütsel Davranış}{Hacettepe Üniversitesi}{}{3.86/4.00}{}
\cventry{2020--2022}{Ön Lisans, Web Tasarımı \& Geliştirme}{Anadolu Üniversitesi}{}{3.08/4.00}{}
\cventry{2013--2017}{Lisans, Psikoloji}{Hacettepe Üniversitesi}{}{3.55/4.00}{}

\end{document}